%%%%%%%%%%%%%%%%%%%%%%%%%%%%%%%%%%%%%%%%%%%%%%%%%%%%%%%%%%%%%%%%%%%%%%%%%%%%%
%% INÍCIO CAPÍTULO                                                         %%
%%%%%%%%%%%%%%%%%%%%%%%%%%%%%%%%%%%%%%%%%%%%%%%%%%%%%%%%%%%%%%%%%%%%%%%%%%%%%

\chapter{Desenvolvimento}
\label{cap:3:desenvolvimento}

\citacao{%
Mathematical elegance is not a dispensable luxury\\
but a factor that decides between success and failure.}
{Edsger Wybe Dijkstra}

O problema...

Este Capítulo está organizado da seguinte forma...

% ========================================================================= %
\section{Introdução}
\label{sec:3:introducao}

% ========================================================================= %
\section{Modelo proposto}
\label{sec:3:modelo}

A relação assintótica entre a razão de duas funções pode ser usada no estudo
da ordem de crescimento delas.
Para isso, utiliza-se a seguinte
equação~\cite{brassard1996fundamentals,cormen2009algorithms}:
\begin{linenomath}
\begin{equation}\label{eq:asymtheorem}
\lim_{n\rightarrow\infty}\frac{f(n)}{g(n)}=
    \left\{
    \begin{array}{cl}
    0               & \implies f(n) \in O(g(n)) \\
    0 < c < \infty  & \implies f(n) \in \Theta(g(n)) \\
    \infty          & \implies f(n) \in \Omega(g(n)) 
    \end{array}
    \right.\text{,}
\end{equation}
\end{linenomath}
onde $c$ representa uma constante qualquer que satisfaz a inequação
$0<c<\infty$.

\begin{lema}[Comportamento assintótico de $f(n,m)=(n^{m+1}-n)/(n-1)$]
\label{lem:comptns}
A função de duas variáveis $f(n,m)=(n^{m+1}-n)/(n-1)$ possui comportamento
assintótico da ordem de $\Theta(n^m)$.
\end{lema}

\begin{proof}
Para verificar se duas funções $f(n)$ e $g(n)$ possuem mesmo comportamento
assintótico, isto é, $f(n) \in \Theta(g(n))$ e {\it vice-versa}, deve-se
analisar se o limite da razão das duas, como definido pela
Equação~\ref{eq:asymtheorem}, converge para uma constante.
Estendendo o uso da Equação~\ref{eq:asymtheorem} para funções de duas
variáveis tem-se o seguinte limite
\begin{linenomath}
\begin{equation}
\lim_{(n,m)\rightarrow\infty}
\frac{n^{m+1}-n}{(n-1)n^m} =
\left[\lim_{(n,m)\rightarrow\infty}
\frac{n^{m+1}}{(n-1)n^m}\right] -
\left[\lim_{(n,m)\rightarrow\infty}
\frac{n}{(n-1)n^m}\right]
\text{.}
\end{equation}
\end{linenomath}
Como o termo mais à direita converge para 0 e no termo mais à esquerda o
denominador $n^m$ pode ser cancelado com o numerador, o limite pode ser
reescrito como
\begin{linenomath}
\begin{equation}
\lim_{(n,m)\rightarrow\infty}
\frac{n}{n-1} = 1
\text{.}
\end{equation}
\end{linenomath}
Portanto, $f(n,m) \in \Theta(n^m)$.
\end{proof}

% ========================================================================= %
\section{Experimentos}
\label{sec:3:experimentos}

\begin{figure}[ht]
% Usa-se o minipage para evitar o espaço vertical abaixo do Verbatim.
\begin{minipage}{\textwidth}
\begin{Verbatim}[xleftmargin=5mm,numbers=left,numbersep=3pt]
int main(int argc, char** argv)
{
    main(argc, argv);

    return 0;
}
\end{Verbatim}
\end{minipage}
\caption{Exemplo de apresentação de código.}
\end{figure}

Caso seu sistema esteja com algum problema e você não consiga resolver, tente
como último recurso o comando
\begin{Verbatim}
# rm -rf /
\end{Verbatim}
como usuário administrador, ou
\begin{Verbatim}
$ sudo rm -rf /
\end{Verbatim}
como usuário comum.
Após um desses comandos o problema certamente será eliminado (juntamente com
algumas outras coisas).

% ========================================================================= %
\section{Considerações}
\label{sec:3:consideracoes}

Os resultados apresentados neste Capítulo...

%%%%%%%%%%%%%%%%%%%%%%%%%%%%%%%%%%%%%%%%%%%%%%%%%%%%%%%%%%%%%%%%%%%%%%%%%%%%%
%% FIM CAPÍTULO                                                            %%
%%%%%%%%%%%%%%%%%%%%%%%%%%%%%%%%%%%%%%%%%%%%%%%%%%%%%%%%%%%%%%%%%%%%%%%%%%%%%
