%%%%%%%%%%%%%%%%%%%%%%%%%%%%%%%%%%%%%%%%%%%%%%%%%%%%%%%%%%%%%%%%%%%%%%%%%%%%%
%% INÍCIO DO ARQUIVO DE COMANDOS                                           %%
%%%%%%%%%%%%%%%%%%%%%%%%%%%%%%%%%%%%%%%%%%%%%%%%%%%%%%%%%%%%%%%%%%%%%%%%%%%%%

% Muda o estilo do cabeçalho dos capítulos so for um relatório
\if\doctype\doctyper           % Se é do tipo relatório
\titleformat{\chapter}{\normalfont\huge\bf}{\thechapter.}{20pt}{\huge\bf}
\fi

\newcommand{\limparpagina}{%
\if\doctype\doctypem           % Se é do tipo monografia
  \cleardoublepage
\fi
\if\doctype\doctyper           % Se é do tipo relatório
  \clearpage
\fi}

% Infenização
\hyphenation{net-works lay-out}

% Comandos próprios
\newcommand*\pct{\scalebox{.9}{\%}}

% Configuração do ambiente Verbatim do fancyvrb
\fvset{fontsize=\small,xleftmargin=1mm,xrightmargin=1mm}

% Configuração de diagramas de Gantt
\ganttset{%
  canvas/.append style={fill=none, dotted},
  %vgrid={draw={none}, dotted},
  vgrid,
  hgrid,
  today label=Prazo,
  group height=.3,
  group left shift=0,
  group right shift=0,
  group peaks height=1,
  group/.style={draw=black,fill=black},    
  bar height=.3,
  bar/.style={fill=black},
  progress label text={#1\pct},
  title height=1}

\newcommand{\citacao}[2]{%
\begin{flushright}
\footnotesize
\parbox{0.95\textwidth}{\flushright{\it``#1''\\ \scriptsize #2}}
\end{flushright}}

\DeclareMathOperator*{\argmin}{arg\,min}
\DeclareMathOperator*{\argmax}{arg\,max}

\newcommand{\definirtitulo}[1]{\newcommand{\titulo}{#1}}
\newcommand{\definirtitulocurto}[1]{\newcommand{\titulocurto}{#1}}

% Algoritmos
\floatname{algorithm}{Algoritmo}
\renewcommand{\listalgorithmname}{Lista de Algoritmos}

\renewcommand{\algorithmicindent}{2.0em}
\renewcommand{\algorithmicrequire}{\textbf{algoritmo}}
\renewcommand{\algorithmicensure}{\textbf{fim algoritmo}}
\renewcommand{\algorithmicreturn}{\textbf{retorne}}
\renewcommand{\algorithmicor}{\textbf{ou}}
\renewcommand{\algorithmicand}{\textbf{e}}
\renewcommand{\algorithmicend}{\textbf{fim}}
\renewcommand{\algorithmicif}{\textbf{se}}
\renewcommand{\algorithmicthen}{\textbf{ent\~ao}}
\renewcommand{\algorithmicfor}{\textbf{para}}
\renewcommand{\algorithmicforall}{\textbf{para cada}}
\renewcommand{\algorithmicrepeat}{\textbf{repita}}
\renewcommand{\algorithmicuntil}{\textbf{enquanto}}
\renewcommand{\algorithmicwhile}{\textbf{enquanto}}
\renewcommand{\algorithmicdo}{\textbf{fa\c{c}a}}

% Cabeçalho e rodapé
\renewcommand{\thefootnote}{[\roman{footnote}]}
\setlength{\headheight}{14.5pt}

\fancypagestyle{labepi}{%
\fancyhead{}
\fancyfoot{}
\if\doctype\doctypem           % Se é do tipo monografia
  \fancyhead[LO]{\bf\nouppercase \rightmark}
  \fancyhead[RO]{\bf\thepage}
  \fancyhead[LE]{\bf\thepage}
  \fancyhead[RE]{\bf\nouppercase \leftmark}
\fi
\if\doctype\doctyper           % Se é do tipo relatório
  \fancyhead[R]{\bf\thepage}
  \fancyhead[L]{\bf\titulocurto}
\fi
\renewcommand{\headrulewidth}{0pt}
\renewcommand{\footrulewidth}{0pt}}

\fancypagestyle{plain}{%
\fancyhead{}
\fancyfoot{}
\fancyhead[R]{\bf\thepage}
\fancyhead[L]{\bf\titulocurto}
\renewcommand{\headrulewidth}{0pt}
\renewcommand{\footrulewidth}{0pt}}

% Teoremas
\declaretheoremstyle[
  spaceabove=6pt,
  spacebelow=6pt,
  headfont=\normalfont\bfseries,
  notefont=\mdseries,
  notebraces={(}{)},
  bodyfont=\normalfont,
  postheadspace=1em,
  qed=$\square$,
  numberwithin=chapter
]{thmsty}

\declaretheorem[style=thmsty,name=Algoritmo]{algoritmo}

\declaretheorem[style=thmsty,name=Definição]{definicao}
\declaretheorem[style=thmsty,name=Premissa]{premissa}

\declaretheorem[style=thmsty,name=Afirmação]{afirmacao}
\declaretheorem[style=thmsty,name=Observação]{observacao}
\declaretheorem[style=thmsty,name=Corolário]{corolario}
\declaretheorem[style=thmsty,name=Lema]{lema}
\declaretheorem[style=thmsty,name=Teorema]{teorema}

\declaretheorem[style=thmsty,name=Nota]{nota}

\renewcommand{\qedsymbol}{{\bf C.Q.D.}}

% Bibliografia
\newcommand{\backrefnotcited}{\relax} %(Not cited.)
\newcommand{\backrefcitedsingle}[1]{\\(Citado na página~#1)}
\newcommand{\backrefcitedmultiple}[1]{\\(Citado nas páginas~#1)}
\renewcommand{\backreftwosep}{ e~}
\renewcommand{\backreflastsep}{ e~}
\renewcommand*{\backref}[1]{}
\renewcommand*{\backrefalt}[4]{%
\ifcase #1
  \backrefnotcited
\or
  \backrefcitedsingle{#2}
\else
  \backrefcitedmultiple{#2}
\fi}

% Índice
\makeindex[intoc,options={-s labepi.ist}]

\hypersetup{colorlinks,linkcolor=blue,citecolor=blue}

\index{recorrencia@recorrência|see{recursividade}}
\index{recursividade|see{recorrência}}

%%%%%%%%%%%%%%%%%%%%%%%%%%%%%%%%%%%%%%%%%%%%%%%%%%%%%%%%%%%%%%%%%%%%%%%%%%%%%
%% FIM DO ARQUIVO DE COMANDOS                                              %%
%%%%%%%%%%%%%%%%%%%%%%%%%%%%%%%%%%%%%%%%%%%%%%%%%%%%%%%%%%%%%%%%%%%%%%%%%%%%%
